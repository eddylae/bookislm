%%%%%%%%%%%%%%%%%%%%%%%%%%%%%%%%%%%%%%%%%
% The Legrand Orange Book
% LaTeX Template
% Version 2.1.1 (14/2/16)
%
% This template has been downloaded from:
% http://www.LaTeXTemplates.com
%
% Original author:
% Mathias Legrand (legrand.mathias@gmail.com) with modifications by:
% Vel (vel@latextemplates.com)
%
% License:
% CC BY-NC-SA 3.0 (http://creativecommons.org/licenses/by-nc-sa/3.0/)
%
% Compiling this template:
% This template uses biber for its bibliography and makeindex for its index.
% When you first open the template, compile it from the command line with the 
% commands below to make sure your LaTeX distribution is configured correctly:
%
% 1) pdflatex main
% 2) makeindex main.idx -s StyleInd.ist
% 3) biber main
% 4) pdflatex main x 2
%
% After this, when you wish to update the bibliography/index use the appropriate
% command above and make sure to compile with pdflatex several times 
% afterwards to propagate your changes to the document.
%
% This template also uses a number of packages which may need to be
% updated to the newest versions for the template to compile. It is strongly
% recommended you update your LaTeX distribution if you have any
% compilation errors.
%
% Important note:
% Chapter heading images should have a 2:1 width:height ratio,
% e.g. 920px width and 460px height.
%
%%%%%%%%%%%%%%%%%%%%%%%%%%%%%%%%%%%%%%%%%

%----------------------------------------------------------------------------------------
%	PACKAGES AND OTHER DOCUMENT CONFIGURATIONS
%----------------------------------------------------------------------------------------

\documentclass[12pt,fleqn]{book} % Default font size and left-justified equations

%----------------------------------------------------------------------------------------

\input{structure} % Insert the commands.tex file which contains the majority of the structure behind the template

\begin{document}

%----------------------------------------------------------------------------------------
%	TITLE PAGE
%-------------------------------------------------------------------------------


\begin{tikzpicture}[remember picture,overlay]
%%%%%%%%%%%%%%%%%%%% Background %%%%%%%%%%%%%%%%%%%%%%%%
\fill[BrickRed] (current page.south west) rectangle (current page.north east);


\foreach \i in {2.5,...,22}
{
    \node[rounded corners,BrickRed!60,draw,regular polygon,regular polygon sides=6, minimum size=\i cm,ultra thick] at ($(current page.west)+(2.5,-5)$) {} ;
}

%%%%%%%%%%%%%%%%%%%% Background Polygon %%%%%%%%%%%%%%%%%%%% 
\foreach \i in {0.5,...,22}
{
\node[rounded corners,BrickRed!60,draw,regular polygon,regular polygon sides=6, minimum size=\i cm,ultra thick] at ($(current page.north west)+(2.5,0)$) {} ;
}

\foreach \i in {0.5,...,22}
{
\node[rounded corners,BrickRed!90,draw,regular polygon,regular polygon sides=6, minimum size=\i cm,ultra thick] at ($(current page.north east)+(0,-9.5)$) {} ;
}


\foreach \i in {21,...,6}
{
\node[BrickRed!85,rounded corners,draw,regular polygon,regular polygon sides=6, minimum size=\i cm,ultra thick] at ($(current page.south east)+(-0.2,-0.45)$) {} ;
}


%%%%%%%%%%%%%%%%%%%% Title of the Report %%%%%%%%%%%%%%%%%%%% 
\node[left,BrickRed!5,minimum width=0.625*\paperwidth,minimum height=3cm, rounded corners] at ($(current page.north east)+(-1,-9.825)$)
{
{\fontsize{25}{30} \selectfont \bfseries MACROECONOMÍA ESTÁNDAR}
};

%%%%%%%%%%%%%%%%%%%% Subtitle %%%%%%%%%%%%%%%%%%%% 
\node[left,BrickRed!10,minimum width=0.625*\paperwidth,minimum height=2cm, rounded corners] at ($(current page.north east)+(-1,-11.3)$)
{
{\huge \textit{Un enfoque algebraico}}
};

%%%%%%%%%%%%%%%%%%%% Author Name %%%%%%%%%%%%%%%%%%%% 
\node[left,BrickRed!5,minimum width=0.625*\paperwidth,minimum height=2cm, rounded corners] at ($(current page.north east)+(-1,-13.3)$)
{
{\Large \textsc{Eddy Lizarazu Alanez}}
};

%%%%%%%%%%%%%%%%%%%% Year %%%%%%%%%%%%%%%%%%%% 
\node[rounded corners,fill=BrickRed!70,text =BrickRed!5,regular polygon,regular polygon sides=6, minimum size=2.5 cm,inner sep=0,ultra thick] at ($(current page.west)+(2.5,-5)$) {\LARGE \bfseries 2025};

\end{tikzpicture}


%----------------------------------------------------------------------------------------
%	COPYRIGHT PAGE
%----------------------------------------------------------------------------------------

\newpage
~\vfill
\thispagestyle{empty}

\noindent Copyright \copyright\ 2025 Eddy Lizarazu\\ % Copyright notice

\noindent \textsc{Published by UAM-Iztapalapa}\\ % Publisher

\noindent \textsc{book-website.com}\\ % URL

\noindent Licensed under the Creative Commons Attribution-NonCommercial 3.0 Unported License (the ``License''). You may not use this file except in compliance with the License. You may obtain a copy of the License at \url{http://creativecommons.org/licenses/by-nc/3.0}. Unless required by applicable law or agreed to in writing, software distributed under the License is distributed on an \textsc{``as is'' basis, without warranties or conditions of any kind}, either express or implied. See the License for the specific language governing permissions and limitations under the License.\\ % License information

\noindent \textit{Primera edición, 2025} % Printing/edition date

%----------------------------------------------------------------------------------------
%	TABLE OF CONTENTS
%----------------------------------------------------------------------------------------

%\usechapterimagefalse % If you don't want to include a chapter image, use this to toggle images off - it can be enabled later with \usechapterimagetrue

\chapterimage{chapter_head_1.pdf} % Table of contents heading image

\pagestyle{empty} % No headers

\tableofcontents % Print the table of contents itself

\cleardoublepage % Forces the first chapter to start on an odd page so it's on the right

\pagestyle{fancy} % Print headers again

%----------------
% Indice
%----------------

%----------------------

%%%%%%%%%%%%%%%%%%%%%%%%%%%%%%%%%%%%%%%%%%%%%
% Macroeconomía IS-LM: Un enfoque algebraico
%
% PARTE I: Introduccion
%   1.	Una panaorámica del libro
%   2.  Indicares económicos y algunas idéntidades
%
% PARTE II: Macroeconomía IS-LM 
%   3.  ⁠La macroeconomía IS-LM
%   4.	El modelo IS-LM keynesiano
%   5.	⁠El modelo IS-LM neoclásico 
%    
% PARTE III: Sector productivo
%   6.	⁠El modelo neoclásico
%   7.	⁠El modelo keynesiano
%
% PARTE IV: Macroeconomía abierta
%   8.  El enfoque de elasticidades y absorción
%   9.	El modelo MF
%   10. El modelo AA-DD de Krugman
%   11. El modelo monetario de economía abierta
%
% PARTE V: Introducción a la dinámica
%   12. El impacto fiscal 
%   13. Acumulación del capital
%   14. ⁠El impacto monetario
%   15. El ataque especulativo
%   16. El modelo de sobrerreacción
%   17: El modelo de la síntesis neoclásica
%   
%
% PARTE VI: Macroeconmía neokeynesiana
%
%   18. Análisis estocástico
%   19. ⁠ER y política económica
%   20. El instrumento monetario
%   21. ⁠Un modelo neokeynesiano de EE 
%   22. ⁠Un modelo neokeynesiano de ER
%
%  APÉNDICES
%   A. La dimensión temporal
%   B. La indeterminación del nivel de precios
%   C. El banco central y el modelo IS-PM
%   D. Fijación de precios y el sector productivo
%   E. Una extensión al modelo SN
%   F. El modelo Barro-Gordon
%   G. Fundamentos de las relaciones agregadas
%   H. xxx


\part{Macroeconomía IS-LM}

%\input{chapters/chapter02/cha02.tex}
\input{chapters/chapter03/cha03.tex}
%\input{chapters/chapter04/cha04.tex}
%\input{chapters/chapter05/cha05.tex}
%\input{chapters/chapter08/cha08.tex}
%\input{chapters/chapter10/cha10.tex}
%\input{chapters/chapter11/cha11.tex}

%\part{Otros modelos dinámicos}

%\input{chapters/chapter12/cha12.tex}
%\input{chapters/chapter13/cha13.tex}
%\input{chapters/chapter14/cha14.tex}


%\part{Expectativas y política económica}


%\input{chapters/chapter15/cha15.tex}
%\input{chapters/chapter16/cha16.tex}
%\input{chapters/chapter17/cha17.tex}
%\input{chapters/chapter18/cha18.tex}


\part{Apéndices}

\appendix
\input{chapters/appendix/apen-A}
%\input{chapters/appendix/apen-B}
%\input{chapters/appendix/apen-C}


\part{Bibliografía}

\input{chapters/biblio/biblio}

%-------%	BIBLIOGRAPHY

%\chapter{Bibliografía}
%\addcontentsline{toc}{chapter}{\textcolor{ocre}{Bibliografía}}
%\section*{Libros}
%\addcontentsline{toc}{section}{Libros}
%\printbibliography[heading=bibempty,type=book]
%\section*{Artículos}
%\addcontentsline{toc}{section}{Articulos}
%\printbibliography[heading=bibempty,type=article]


%\part*{Indice de palabras}

%------------------------------------------------
%	INDEX
%------------------------------------------------

%\cleardoublepage
%\phantomsection
%\setlength{\columnsep}{0.75cm}
%\addcontentsline{toc}{chapter}{\textcolor{ocre}{Indice de palabras}}
%\printindex

\end{document}
